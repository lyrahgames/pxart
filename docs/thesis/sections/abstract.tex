\documentclass[crop=false]{standalone}
\usepackage{standard}
\begin{document}
  \newgeometry{top=0mm,bottom=0mm,right=43mm,left=43mm}
  {
    \thispagestyle{empty}
    \null
    \vfill
    \hrule
    \medskip
    \begin{abstract}
      \itshape
      The topic of this thesis is the development of a software library, called \citetitle{pxart}, in the C++ programming language.
      It makes it possible to improve the performance of physical simulations that need to use random numbers.
      The library consists of pseudorandom number generators exploiting the SSE and AVX instruction set facilities of modern Intel processors.
      For this, a new interface was created to simplify their initialization and usage.
      % Through the implementation of appropriate benchmarks and physically-based applications, like photon propagation and the Ising model, we were able to show that our generators could indeed accelerate the execution of randomized algorithms.
      Through the implementation of appropriate benchmarks and the simulation of photon propagation it was shown that the vectorized generators could indeed accelerate the execution of randomized algorithms.
    \end{abstract}
    \medskip
    \hrule
    \vfill
  }
  \restoregeometry
\end{document}