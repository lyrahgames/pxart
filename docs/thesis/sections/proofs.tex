\documentclass{stdlocal}
\begin{document}
\section{Mathematical Proofs} % (fold)
\label{sec:proofs}
  \begin{lemma*}[Monte Carlo Integration Estimates Value of Integral]
    Choose the same setting as in the above definition \ref{definition:monte-carlo-integration}.
    In this case for all $n\in\setNatural$, the Monte Carlo integration $\mathrm{MCI}_n(f)$ is a Monte Carlo method and the following statements for the expectation value and standard deviation are fulfilled.
    \[
      \expect \mathrm{MCI}_n(f) = \integral{U}{}{f}{λ}
      \separate
      \stddev\boxBrackets{\mathrm{MCI}_n(f)} \leq \sqrt{\frac{λ(U)}{n} \integral{U}{}{f^2}{λ}}
    \]
  \end{lemma*}
  \begin{proof}[Lemma \ref{lemma:monte-carlo-integration} on page \pageref{lemma:monte-carlo-integration}]
    Let $p$ be the probability density of $X_n$.
    Because the random variables are uniformly distributed on $U$, we can express it as follows.
    \[
      \function{p}{U}{[0,\infty)}
      \separate
      p(x) \define \frac{1}{λ(U)}
    \]
    By using substitution and chaining from propositions \ref{proposition:substitution} and \ref{proposition:chaining}, the expectation value can be directly computed.
    \[
      \begin{aligned}[t]
        \expect \mathrm{MCI}_n(f)
        &= \expect \boxBrackets{ \frac{λ(U)}{n} \sum_{k=1}^n f\circ X_k }
        = \frac{λ(U)}{n} \sum_{k=1}^n \expect(f\circ X_k) \\
        &= λ(U) \integral{U}{}{f(x) p(x)}{λ(x)}
        = \integral{U}{}{f}{λ}
      \end{aligned}
    \]
    For the standard deviation, first the variance will be observed.
    Since the sequence of random variables is stochastically independent, the sum can be taken out of the argument.
    Afterwards, we again apply substitution and chaining.
    \[
      \begin{aligned}
        \var \mathrm{MCI}_n(f) &= \var\boxBrackets{ \frac{λ(U)}{n} \sum_{k=1}^n f\circ X_k } = \frac{λ(U)^2}{n^2} \sum_{k=1}^n \var\roundBrackets{f\circ X_k} \\
        &= \frac{λ(U)^2}{n^2} \sum_{k=1}^n \expect\roundBrackets{f\circ X_k}^2 - \boxBrackets{\expect\roundBrackets{f\circ X_k}}^2 \\
        &\leq \frac{λ(U)^2}{n^2} \sum_{k=1}^n \expect\roundBrackets{f\circ X_k}^2 = \frac{λ(U)^2}{n} \integral{U}{}{f^2(x) p(x)}{λ(x)} \\
        &= \frac{λ(U)}{n} \integral{U}{}{f^2}{λ}
      \end{aligned}
    \]
    The inequality is now inferred by the definition of the standard deviation which proofs the lemma.
    \[
      \stddev\boxBrackets{\mathrm{MCI}_n(f)} = \sqrt{\var \mathrm{MCI}_n(f)} \leq \sqrt{\frac{λ(U)}{n} \integral{U}{}{f^2}{λ}}
    \]
  \end{proof}

  \begin{lemma*}[Pseudorandom Sequences are Ultimately Periodic]
    Let $\mathscr{G}\define (S,T,U,G)$ be a PRNG and $s_0\in S$ its initial state.
    Then the respective pseudorandom sequence $(u_n)_{n\in\setNatural}$ is ultimately periodic.
    In this case, for the period ρ and the transient τ the following holds.
    \[
      1 \leq ρ + τ - 1 \leq \# S
    \]
    In particular, if $T$ is bijective $(u_n)$ will be periodic.
  \end{lemma*}
  \begin{proof}[Lemma~\ref{lemma:pseudorandom-sequences-periodicity} on page \pageref{lemma:pseudorandom-sequences-periodicity}]
    Let $(s_n)_{n\in\setNatural}$ be the respective sequence of states and $N\define \# S$ the number of different states.
    $T$ maps all elements of $S$ to at most $N$ other elements of $S$.
    Therefore at least the element $s_N$ has to be mapped to an element $s_k$ for $k\in\setNatural$ with $k\leq N$ which was already reached.
    % Hence, there exist $n,k\in\setNatural$ with $k\leq n\leq N$ such that $T(s_n) = s_k$.
    Hence, we conclude the following.
    \[
      \exists n,k\in\setNatural, k\leq n\leq N: \quad T(s_n) = s_k
    \]
    % Assume $T$ maps $s_n$ to a state $s_k$ with $k\in\setNatural$ and $k < n$.
    We choose $n$ and $k$ appropriately and define the following values.
    \[
      ρ \define n - k + 1
      \separate
      τ \define k
    \]
    Now let $i\in\setNatural_0$ be arbitrary and apply the definition.
    We get the following chain of equations which show that $(u_n)$ is ultimately periodic.
    \[
      \begin{aligned}
        u_{τ+i+ρ} &= u_{n+1+i} = G \circ T^{n+1+i}(s_0) = G \circ T^i\circ T^{n+1}(s_0) \\
        &= G \circ T^i(s_k) = G \circ T^i \circ T^k(s_0) = G \circ T^{i+k}(s_0) = u_{k+i} = u_{τ+i}
      \end{aligned}
    \]
    The inequality can be shown by directly inserting the values into the definition.
    \[
      1 \leq ρ + τ - 1 = n \leq N = \# S
    \]
    This proofs the given lemma.
  \end{proof}

  \begin{lemma*}[Equidistributed Pseudorandom Sequences]
    Let $\mathscr{G}\define (S,T,U,G)$ be a PRNG with $s_0\in S$ as its seed value and $(u_n)_{n\in\setNatural}$ the respective pseudorandom sequence with transient τ and period ρ.
    Furthermore, let μ be a probability measure on $(U,\mathscr{P}(U))$.
    Then the following statements are equivalent.
    \begin{enumerate}[label=(\roman*)]
      \item $(u_n)$ is equidistributed with respect to μ.
      \item For all $u\in U$ the following is true.
        \[
          \frac{1}{ρ} \cdot \#\set{n\in\setNatural}{τ\leq n < ρ+τ, u_n = u} = μ(\set{u}{})
        \]
    \end{enumerate}
  \end{lemma*}
  \begin{proof}[Lemma~\ref{lemma:equidistribution} on page \pageref{lemma:equidistribution}]
    Because $U$ is a finite set, every measurable function $\function{X}{U}{\setReal}$ can be described as a linear combination of characteristic functions with respect to some real coefficients $α_u$ for all $u \in U$ in the following way.
    \[
      X = \sum_{u\in U} α_u \mathds{1}_{\set{u}{}}
    \]
    Hence, without loss of generality, it suffices to take only characteristic functions into account.
    Let $u\in U$ be arbitrary.
    The right-hand side of the definition will then result in the following.
    \[
      \integral{U}{}{\mathds{1}_{\set{u}{}}}{μ} = μ(\set{u}{})
    \]
    Applying the characteristic function together with the properties of a periodic sequence to the left-hand side of the definition, looks as follows.
    \[
      \begin{aligned}
        \lim_{n\to\infty} \frac{1}{n} \sum_{k=1}^n \mathds{1}_{\set{u}{}}(u_k)
        &= \lim_{n\to\infty} \frac{1}{n} \sum_{k=1}^{τ-1} \mathds{1}_{\set{u}{}}(u_k) + \lim_{n\to\infty} \frac{1}{n}\sum_{k=τ}^{τ+n-1} \mathds{1}_{\set{u}{}}(u_k) \\
        &= \frac{1}{ρ} \sum_{k=τ}^{τ+ρ-1} \mathds{1}_{\set{u}{}}(u_k) \\
        &= \frac{1}{ρ} \cdot \#\set{n\in\setNatural}{τ\leq n < ρ+τ, u_n = u}
      \end{aligned}
    \]
    This shows the desired equivalence and proofs the lemma.
  \end{proof}

  \begin{lemma*}[Corresponding Vector Sequences are Ultimately Periodic]
    Let $U$ be a non-empty set of values and $(u_n)_{n\in\setNatural}$ be an ultimately periodic sequence in $U$ with period ρ and transient τ.
    In this case, every corresponding $k$-dimensional vector sequence $(v_n)_{n\in\setNatural}$ with translation $t$ is ultimately periodic with period $ρ'$ and transient $τ'$ defined as follows.
    \[
      ρ' \define \frac{ρ}{\mathrm{gcd}(ρ,k)}
      \separate
      τ' \define \ceilBrackets{\frac{\max(0,τ-1-t)}{k}} + 1
    \]
  \end{lemma*}
  \begin{proof}[Lemma \ref{lemma:vector-sequences-periodicity} on page \pageref{lemma:vector-sequences-periodicity}]
    Choose $n\in\setNatural_0$ and $i\in\setNatural$ with $i\leq k$ to be arbitrary.
    We denote with $v^{(i)}_n$ the $i$.~coordinate of the $n$.~vector.
    By definition the following equality holds.
    \[
      v^{(i)}_{τ' + n + ρ'} = u_{t + (τ'+n+ρ'-1)k + i}
    \]
    Observing the index, we separate it into three parts.
    One for the index, one for the transient one for the period.
    \[
      t+(τ' + n + ρ' - 1)k + i = \underbrace{(t + τ'k - k + 1)}_{\reverseDefine \tilde{τ}} + \underbrace{(nk + i - 1)}_{\reverseDefine \tilde{n}}  + \underbrace{ρ'k}_{\reverseDefine \tilde{ρ}}
    \]
    The period part has to be a multiple of the period ρ of $(u_n)$ as can be seen in the following.
    Hence, $\tilde{ρ}$ has the property of a period.
    \[
      \tilde{ρ} = ρ'k = \frac{ρk}{\mathrm{gcd}(ρ,k)} = ρ \frac{k}{\mathrm{gcd}(ρ,k)}
    \]
    To apply the periodicity of $(u_n)$, the transient part has to be bigger or equal to the transient τ of $(u_n)$.
    \[
      \tilde{τ} = t + τ'k - k + 1 = 1 + t + k \ceilBrackets{\frac{\max(0,τ-1-t)}{k}} \geq τ
    \]
    Inserting the results and applying the periodicity of $(u_n)$, we can conclude that the corresponding vector sequence has to be ultimately periodic as well.
    \[
      v^{(i)}_{τ' + n + ρ'} = u_{\tilde{τ} + \tilde{n} + \tilde{ρ}} = u_{\tilde{τ} + \tilde{n}} = u_{t + (τ' + n - 1)k + i} = v^{(i)}_{τ' + n}
    \]
    Due to the shown statements, $ρ'$ and $τ'$ are indeed the smallest possible values such that this equation holds and can therefore be denoted as period and transient of $(v_n)$ respectively.
  \end{proof}
% section proofs (end)
\end{document}