\documentclass[crop=false]{stdlocal}
\begin{document}
\section{Conclusions and Future Work} % (fold)
\label{sec:conclusions}
  We were able to show that the vectorization of PRNGs based on SIMD intrinsics can improve the performance of physical simulations which need to access a large amount of random numbers.
  For further development, we recommend to compare our implementations against state-of-the-art RNGs.
  Due to the unique restrictions concerning the statistical performance an application is imposing on an RNG, future work should involve the development of even more vectorized RNGs.
  This also means that we have to exhaustively test the statistical performance of vectorized PRNGs in different scenarios.
  These scenarios should be simplified versions of a real-world problem to justify the results and examine the robustness of our current design.
  By tweaking the testing and benchmarking framework, we are then able to extend the vectorized implementations to other SIMD architectures providing vector registers, like the AVX512 instruction set from Intel.
  With the release of the C++20 standard specification, we should take advantage on the newly introduced features, like concepts, of C++ to reduce the complexity of the API and increase its performance and handling.

  more generators
  comparison to other implementations
  c++20 improve api
  real-world applications
  avx512
  more architectures
  statistical tests
% section conclusions (end)
\end{document}