\documentclass[crop=false]{standalone}
\usepackage{standard}
\begin{document}
  \newgeometry{top=0mm,bottom=0mm,right=43mm,left=43mm}
  {
    \selectlanguage{ngerman}
    \thispagestyle{empty}
    \null
    \vfill
    \hrule
    \medskip
    \begin{abstract}
      \itshape
      Das Thema dieser Arbeit ist die Entwicklung einer Softwarebibliothek, genannt \enquote{pxart}, in der Programmiersprache C++, die es ermöglicht, die Leistung auf Zufallszahlen basierender physikalischer Simulationen zu verbessern.
      Die Bibliothek besteht aus Pseudozufallszahlengeneratoren, die die SSE- und AVX-Befehlssätze moderner Intel Prozessoren nutzen.
      Wir haben eine neue Schnittstelle geschaffen, um deren Initialisierung und Nutzung zu vereinfachen.
      Durch die Implementierung geeigneter Benchmarks und physikalisch basierter Applikationen konnten wir zeigen, dass unsere Generatoren die Ausführung randomisierter Algorithmen tatsächlich beschleunigen.
    \end{abstract}
    \medskip
    \hrule
    \vfill
  }
  \restoregeometry
\end{document}