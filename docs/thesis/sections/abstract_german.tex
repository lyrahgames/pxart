\documentclass[crop=false]{standalone}
\usepackage{standard}
\begin{document}
  \newgeometry{top=0mm,bottom=0mm,right=43mm,left=43mm}
  {
    \selectlanguage{ngerman}
    \thispagestyle{empty}
    \null
    \vfill
    \hrule
    \medskip
    \begin{abstract}
      \itshape
      Das Thema dieser Arbeit ist die Entwicklung einer Softwarebibliothek, genannt \citetitle{pxart}, in der Programmiersprache C++.
      Diese ermöglicht es, die Leistung auf Zufallszahlen basierender physikalischer Simulationen zu verbessern.
      Die Bibliothek besteht aus Pseudozufallszahlengeneratoren, die die SSE- und AVX-Befehlssätze moderner Intel Prozessoren nutzen.
      Es wurde eine neue Schnittstelle geschaffen, um deren Initialisierung und Nutzung zu vereinfachen.
      Durch die Implementierung geeigneter Benchmarks und der Simulation der Photonenausbreitung konnte gezeigt werden, dass die vektorisierten Generatoren die Ausführung randomisierter Algorithmen tatsächlich beschleunigen.
    \end{abstract}
    \medskip
    \hrule
    \vfill
  }
  \restoregeometry
\end{document}