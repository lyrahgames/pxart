\documentclass{stdlocal}
\begin{document}
\section{Previous Work} % (fold)
\label{sec:previous_work}
  % The topic of PRNGs consists of several smaller parts.
  % From a mathematical point of view, one has to talk about their definition and construction as well as methods on how to test their randomness.
  % There have been a lot of publications concerning these issues.
  % Hence, I am not able to give you a detailed overview.
  % Instead, I will focus on the most relevant PRNGs and test suites, as well as some modern examples.

  % The creation of new PRNGs is sometimes understood to be black magic and can be hard since basically, one has to build a deterministic algorithm with a nearly non-deterministic output.
  % In \textcite{kneusel2018} one can find numerous different families of PRNGs.
  % The most well-known ones are Linear Congruential Generators, Mersenne Twisters and Xorshift with its Variants.
  % Whereas LCGs tend to be fast but weak generators in \textcite{oneill2014}, one can find a further developed promising family of algorithms, called PCGs.
  % \textcite{widynski2019} describes another RNG based on the so-called middle square Weyl sequence.
  % All of these generators have certain advantages and disadvantages in different areas such as security, games, and simulations.

  % After building a PRNG, one has to check if the generated sequence of random numbers fulfills certain properties.
  % In general, these properties will somehow measure the randomness of our RNG.
  % Typically, there are a lot of tests bundled inside a test suite such as TestU01 and Dieharder.
% section previous_work (end)
\end{document}