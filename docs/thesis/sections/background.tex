\documentclass{stdlocal}
\begin{document}
\section{Background} % (fold)
\label{sec:background}
  % virtual subsection Introduction
  To systematically approach the implementation of PRNGs, basic knowledge in the topics of stochastics and finite fields is administrable.
  Together, these topics will give a deeper understanding of randomness in deterministic computer systems, a formal description of pseudorandom sequences and generators, and the mathematical foundation of Monte Carlo algorithms.
  Based on them, we are capable of scientifically analyzing PRNGs concerning their randomness properties.
  Vectorization techniques can be conceptualized by the architecture of modern SIMD-capable multiprocessors and their instruction sets.
  Especially the knowledge of typical instructions will make the design of a new API and its application to Monte Carlo simulations clear.
  The following sections will give an overview of the named topics.

  \subsection{Mathematical Preliminaries} % (fold)
  \label{sub:mathematical_preliminaries}
    \subsubsection*{Stochastics} % (fold)
    \label{ssub:stochastics}
      The observation of random processes resulted in the construction of probability theory.
      But an introduction to it can be given without a further formalized concept of randomness.
      Randomness itself plays a minor role and is used in form of realizations of random variables.
      As a consequence probability theory allows us to observe randomness without defining it.
      Actually, typical formalizations rely on probability theory.
      This connection makes the development of RNGs possible.
      Hence, in the following we will give only the formal definition of relevant structures without further discussions and will postpone an examination of randomness to the next subsection.

      \begin{definition}[Probability Space]
        A tuple $\roundBrackets{\Omega, \mathscr{A}, P}$ with a non-empty set $\Omega$, a $σ$-Algebra $\mathscr{A}$, and probability measure $P$ is called probability space if the following conditions hold.
        \begin{enumerate}
          \item $\emptyset, \Omega \in \mathscr{A}$
          \item $A \in \mathscr{A} \implies \Omega\setminus A \in \mathscr{A}$
          \item $\set{A_n}{n\in\setNatural} \subset \mathscr{A} \implies \bigcup_{n\in\setNatural} A_n \in \mathscr{A}$
        \end{enumerate}

        \begin{enumerate}
          \item $P(\Omega) = 1$
          \item $\set{A_n}{n\in\setNatural} \subset \mathscr{A}, A_n\cap A_m = \emptyset \implies P(\bigcup_{n\in\setNatural} A_n) = \sum_{n\in\setNatural} P(A_n)$
        \end{enumerate}
      \end{definition}

      \begin{definition}[Random Variable]
        A random variable is a measurable function $\function{X}{\Omega}{\setReal}$ on a probability space.
        We call $X(ω)$ for $ω\in\Omega$ a realization of $X$.
      \end{definition}

      \begin{definition}[Independence]

      \end{definition}

      \begin{definition}[Probability Measure of a Random Variable]
        \[
          P_X = P \composition \inverse{X}
        \]
      \end{definition}

      \begin{definition}[Expectation Value]
        \[
          \mathds{E}X = \integral{\Omega}{}{X}{P}
        \]
      \end{definition}

      \begin{definition}[Probability Density and Cumulative Distribution Function]
        \[
          P_X(A) = \integral{A}{}{p(x)}{λ(x)}
        \]
      \end{definition}

      \begin{proposition}
        \[
          \mathds{E}f(X) = \integral{\Omega}{}{f(x)p(x)}{λ(x)}
        \]
      \end{proposition}
    % subsubsection stochastics (end)

    \subsubsection*{Finite Fields} % (fold)
    \label{ssub:finite_fields}

    % subsubsection finite_fields (end)
  % subsection mathematical_preliminaries (end)

  \subsection{Pseudorandom Number Generators} % (fold)
  \label{sub:pseudorandom_number_generators}
    PRNGs were first introduced by Neumann.
    A further discussion about randomness will make clear why the design of PRNGs can be seen as art.

    \subsubsection*{Random and Pseudorandom Sequences} % (fold)
    \label{ssub:random_and_pseudorandom_sequences}
      Originating in gambling and physical processes, randomness is a difficult concept and drives many philosophical discussions.
      Typically humans have a bad idea of randomness.
      Randomness makes only sense when it is applied to a sequence of values.
      Because we want to generate random numbers we only need a formal mathematical structure to work with.
      However, a widely accepted unique formal concept has not been found.
      But as stated in \cite{volchan2002} the question if a sequence is random decides at infinity.
      A random sequence, in general, is not computable or compressible by an algorithm.
      Even the methods to test this kind of randomness cannot be computed.
      For the development of RNGs on a computer, we cannot use such concepts.
      A computer in our sense is only capable of using finite sequences and has to compute its randomness to check it.
      Therefore, again as stated in \cite{volchan2002}, we will stick to 'if it looks random, it is random'.
      In \cite{kneusel2018} the concept of randomness was simplified.
    % subsubsection* random_and_pseudorandom_sequences (end)

    \subsubsection*{Random and Pseudorandom Number Generators} % (fold)
    \label{ssub:random_and_pseudorandom_number_generators}
      \begin{definition}[Pseudorandom Number Generator]
        A tuple $(S,s_0,T,U,G)$ is called a PRNG.
        $S$ is a non-empty, finite set of states.
        $s_0 \in S$ is the initial state.
        $\function{T}{S}{S}$ is the transition function.
        $U$ is a non-empty, finite set of output symbols.
        $\function{G}{S}{U}$ is the output function which generates an output symbol for every state.
      \end{definition}

      \begin{definition}[Pseudorandom Sequence of PRNG]
        $(s_n)_{n\in\setNatural}$ is the respective sequence of states
        \[
          s_{n+1} \define T(s_n)
        \]
        Pseudorandom sequence $(u_n)_{n\in\setNatural}$
        \[
          u_n = G(s_n)
        \]
      \end{definition}

      \begin{alignat*}{3}
        s_0 \xrightarrow{T} &s_1 \xrightarrow{T} &&s_2 \xrightarrow{T} &&\ldots \\
        G &\downarrow &&\downarrow \\
        &u_1 &&u_2 &&\ldots
      \end{alignat*}
    % subsubsection* random_and_pseudorandom_number_generators (end)
  % subsection pseudorandom_number_generators (end)

  \subsection{SIMD-Capable Processors} % (fold)
  \label{sub:simd-capable_processors}
    \subsubsection*{Architecture of Modern Central Processing Units} % (fold)
    \label{ssub:architecture_of_modern_central_processing_units}

    % subsubsection* architecture_of_modern_central_processing_units (end)

    \subsubsection*{SIMD Instruction Sets and Efficiency} % (fold)
    \label{ssub:simd_instruction_sets_and_efficiency}

    % subsubsection* simd_instruction_sets_and_efficiency (end)

    \subsubsection*{SSE, AVX, AVX512} % (fold)
    \label{ssub:sse_avx_avx512}

    % subsubsection* sse_avx_avx512 (end)
  % subsection simd-capable_processors (end)

  \subsection{Simulation in Physics and Mathematics} % (fold)
  \label{sub:simulation_in_physics_and_mathematics}
    \subsubsection*{Mathematical and Physical Preliminaries} % (fold)
    \label{ssub:mathematical_and_physical_preliminaries}

    % subsubsection* mathematical_and_physical_preliminaries (end)

    \subsubsection*{Baseline Model Problems} % (fold)
    \label{ssub:baseline_model_problems}

    % subsubsection* baseline_model_problems (end)
  % subsection simulation_in_physics_and_mathematics (end)

  \subsection{Summary} % (fold)
  \label{sub:summary}

  % subsection summary (end)
% section background (end)
\end{document}