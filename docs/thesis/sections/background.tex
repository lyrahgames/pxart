\documentclass{stdlocal}
\begin{document}
\section{Background} % (fold)
\label{sec:background}
  % virtual subsection Introduction
  To systematically approach the implementation of PRNGs, basic knowledge in the topics of stochastics and finite fields is administrable.
  Together, these topics will give a deeper understanding of randomness in deterministic computer systems, a formal description of pseudorandom sequences and generators, and the mathematical foundation of Monte Carlo algorithms.
  Based on such concepts, we are capable of scientifically analyzing PRNGs concerning their randomness properties.
  Vectorization techniques can be conceptualized by the architecture of modern SIMD-capable multiprocessors and their instruction sets.
  Especially the knowledge of typical instructions will make the design of a new API and its application to Monte Carlo Simulations clear.
  The following sections will give an overview of the named topics.

  \subsection{Mathematical Preliminaries} % (fold)
  \label{sub:mathematical_preliminaries}
    \subsubsection*{Randomness} % (fold)
    \label{ssub:randomness}

    % subsubsection randomness (end)

    \subsubsection*{Stochastics} % (fold)
    \label{ssub:stochastics}
      \begin{definition}[Probability Space]
        A tuple $\roundBrackets{\Omega, \mathscr{A}, P}$ with a non-empty set $\Omega$, a $σ$-Algebra $\mathscr{A}$, and probability measure $P$ is called probability space if the following conditions hold.
      \end{definition}

      \begin{definition}[Random Variable]

      \end{definition}

      \begin{definition}[Independence]

      \end{definition}

      \begin{definition}[Probability Measure of a Random Variable]

      \end{definition}

      \begin{definition}[Expectation Value]

      \end{definition}

      \begin{definition}[Probability Density and Cumulative Distribution Function]

      \end{definition}
    % subsubsection stochastics (end)

    \subsubsection*{Finite Fields} % (fold)
    \label{ssub:finite_fields}

    % subsubsection finite_fields (end)
  % subsection mathematical_preliminaries (end)

  \subsection{Pseudorandom Number Generators} % (fold)
  \label{sub:pseudorandom_number_generators}
    \subsubsection*{Mathematical Preliminaries} % (fold)
    \label{ssub:mathematical_preliminaries}

    % subsubsection* mathematical_preliminaries (end)

    \subsubsection*{Random and Pseudorandom Sequences} % (fold)
    \label{ssub:random_and_pseudorandom_sequences}

    % subsubsection* random_and_pseudorandom_sequences (end)

    \subsubsection*{Random and Pseudorandom Number Generators} % (fold)
    \label{ssub:random_and_pseudorandom_number_generators}

    % subsubsection* random_and_pseudorandom_number_generators (end)

    \subsubsection*{Baseline Examples} % (fold)
    \label{ssub:baseline_examples}

    % subsubsection* baseline_examples (end)
  % subsection pseudorandom_number_generators (end)

  \subsection{SIMD-Capable Processors} % (fold)
  \label{sub:simd-capable_processors}
    \subsubsection*{Architecture of Modern Central Processing Units} % (fold)
    \label{ssub:architecture_of_modern_central_processing_units}

    % subsubsection* architecture_of_modern_central_processing_units (end)

    \subsubsection*{SIMD Instruction Sets and Efficiency} % (fold)
    \label{ssub:simd_instruction_sets_and_efficiency}

    % subsubsection* simd_instruction_sets_and_efficiency (end)

    \subsubsection*{SSE, AVX, AVX512} % (fold)
    \label{ssub:sse_avx_avx512}

    % subsubsection* sse_avx_avx512 (end)
  % subsection simd-capable_processors (end)

  \subsection{Simulation in Physics and Mathematics} % (fold)
  \label{sub:simulation_in_physics_and_mathematics}
    \subsubsection*{Mathematical and Physical Preliminaries} % (fold)
    \label{ssub:mathematical_and_physical_preliminaries}

    % subsubsection* mathematical_and_physical_preliminaries (end)

    \subsubsection*{Baseline Model Problems} % (fold)
    \label{ssub:baseline_model_problems}

    % subsubsection* baseline_model_problems (end)
  % subsection simulation_in_physics_and_mathematics (end)

  \subsection{Summary} % (fold)
  \label{sub:summary}

  % subsection summary (end)
% section background (end)
\end{document}