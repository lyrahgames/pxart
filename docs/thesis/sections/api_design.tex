\documentclass{stdlocal}
\begin{document}
\section{Design of the API} % (fold)
\label{sec:design_of_the_api}
  \inputCodeBlock[title = Is-Valid Utility]{code/is_valid.hpp}
  \inputCodeBlock[title = Uniform Template]{code/uniform.hpp}

  What do we want from the interface of our RNG?
  It should make testing with given frameworks like TestU01, dieharder, ent and PractRand easy.
  Benchmarking should be possible as well.
  Therefore we need a good API and a good application interface.
  Most of the time we want to generate uniform distributed real or integer numbers.
  We need two helper functions.
  So we see that the concept of a distribution makes things complicated.
  We cannot specialize distributions for certain RNGs.
  We cannot use lambda expressions as distributions.
  Therefore we want to use only helper functions as distributions and not member functions.
  So we do not have to specify a specialization and instead use the given standard but we are able to do it.
  Therefore functors and old-distributions are distributions as well and hence we are compatible to the standard.

  Additionally, we have to be more specific about the concept of a random number engine.
  The output of a random number engine of the current concept is magical unsigned integer which should be uniformly distributed in the interval [min,max].
  But these magic numbers can result in certain problems if used the wrong way, see Melissa O'Neill Seeding Surprises.
  Therefore the general idea is to always use the helper functions as new distributions which define min and max explicitly and make sure you really get those values.
  This is also a good idea for the standard.
  And it is compatible with the current standard.

  Now think of vector registers and multiprocessors.
  The random number engine should provide ways to fill a range with random numbers such that it can perform generation more efficiently.
  Think about the execution policies in C++17.
  They should be provided as well.
% section design_of_the_api (end)
\end{document}